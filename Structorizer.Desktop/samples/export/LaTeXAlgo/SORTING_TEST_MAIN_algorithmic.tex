\documentclass[a4paper,10pt]{article}

\usepackage{algorithm}
\usepackage{algorithmic}
\usepackage{ngerman}
\usepackage{amsmath}

\DeclareMathOperator{\oprdiv}{div}
\DeclareMathOperator{\oprshl}{shl}
\DeclareMathOperator{\oprshr}{shr}
\title{Structorizer LaTeX pseudocode Export of SORTING\_TEST.arrz}
\author{Kay G"urtzig}
\date{02.11.2021}

\begin{document}

\floatname{algorithm}{Procedure}
\begin{algorithm}
\caption{bubbleSort(values)}
\begin{algorithmic}[5]

\STATE \COMMENT{ Implements the well-known BubbleSort algorithm. }
\STATE \COMMENT{ Compares neigbouring elements and swaps them in case of an inversion. }
\STATE \COMMENT{ Repeats this while inversions have been found. After every }
\STATE \COMMENT{ loop passage at least one element (the largest one out of the }
\STATE \COMMENT{ processed subrange) finds its final place at the end of the }
\STATE \COMMENT{ subrange. }
  \STATE \(ende\gets\ length(values)-2\)
  \REPEAT
    \STATE \(posSwapped\gets-1\)
    \COMMENT{The index of the most recent swapping (-1 means no swapping done).}
    \FOR{\(i \gets 0\) \TO \(ende\) \textbf{by} \(1\)}
      \IF{\(values[i]>values[i+1]\)}
        \STATE \(temp\gets\ values[i]\)
        \STATE \(values[i]\gets\ values[i+1]\)
        \STATE \(values[i+1]\gets\ temp\)
        \STATE \(posSwapped\gets\ i\)
      \ENDIF
    \ENDFOR
    \STATE \(ende\gets\ posSwapped-1\)
  \UNTIL{\(posSwapped<0\)}

\end{algorithmic}
\end{algorithm}


\floatname{algorithm}{Procedure}
\begin{algorithm}
\caption{maxHeapify(heap, i, range)}
\begin{algorithmic}[5]

\STATE \COMMENT{ Given a max-heap '{}heap\textasciiacute{} with element at index '{}i\textasciiacute{} possibly }
\STATE \COMMENT{ violating the heap property wrt. its subtree upto and including }
\STATE \COMMENT{ index range-1, restores heap property in the subtree at index i }
\STATE \COMMENT{ again. }
  \STATE \(right\gets(i+1)*2\)
  \COMMENT{Indices of left and right child of node i}
  \STATE \(left\gets\ right-1\)
  \STATE \(max\gets\ i\)
  \COMMENT{Index of the (local) maximum}
  \IF{\(left<range\ \AND\ heap[left]>heap[i]\)}
    \STATE \(max\gets\ left\)
  \ENDIF
  \IF{\(right<range\ \AND\ heap[right]>heap[max]\)}
    \STATE \(max\gets\ right\)
  \ENDIF
  \IF{\(max\neq\ i\)}
    \STATE \(temp\gets\ heap[i]\)
    \STATE \(heap[i]\gets\ heap[max]\)
    \STATE \(heap[max]\gets\ temp\)
    \STATE \(maxHeapify(heap,max,range)\)
  \ENDIF

\end{algorithmic}
\end{algorithm}


\floatname{algorithm}{Function}
\begin{algorithm}
\caption{partition(values, start, stop, p)}
\begin{algorithmic}[5]

\STATE \COMMENT{ Partitions array '{}values\textasciiacute{} between indices '{}start\textasciiacute{} und '{}stop\textasciiacute{}-1 with }
\STATE \COMMENT{ respect to the pivot element initially at index '{}p\textasciiacute{} into smaller }
\STATE \COMMENT{ and greater elements. }
\STATE \COMMENT{ Returns the new (and final) index of the pivot element (which }
\STATE \COMMENT{ separates the sequence of smaller elements from the sequence }
\STATE \COMMENT{ of greater elements). }
\STATE \COMMENT{ This is not the most efficient algorithm (about half the swapping }
\STATE \COMMENT{ might still be avoided) but it is pretty clear. }
  \STATE \(pivot\gets\ values[p]\)
  \COMMENT{Cache the pivot element}
  \STATE \(values[p]\gets\ values[start]\)
  \COMMENT{Exchange the pivot element with the start element}
  \STATE \(values[start]\gets\ pivot\)
  \STATE \(p\gets\ start\)
  \STATE \(start\gets\ start+1\)
  \COMMENT{Beginning and end of the remaining undiscovered range}
  \STATE \(stop\gets\ stop-1\)
  \STATE \COMMENT{ Still unseen elements? }
  \STATE \COMMENT{ Loop invariants: }
  \STATE \COMMENT{ 1. p = start - 1 }
  \STATE \COMMENT{ 2. pivot = values[p] }
  \STATE \COMMENT{ 3. i \textless start \textrightarrow{} values[i] ≤ pivot }
  \STATE \COMMENT{ 4. stop \textless i \textrightarrow{} pivot \textless values[i] }
  \WHILE{\(start\leq\ stop\)}
    \STATE \(seen\gets\ values[start]\)
    \COMMENT{Fetch the first element of the undiscovered area}
    \IF{\(seen\leq\ pivot\)}[Does the checked element belong to the smaller area?]
      \STATE \(values[p]\gets\ seen\)
      \COMMENT{Insert the seen element between smaller area and pivot element}
      \STATE \(values[start]\gets\ pivot\)
      \STATE \COMMENT{ Shift the border between lower and undicovered area, }
      \STATE \COMMENT{ update pivot position. }
      \STATE \(p\gets\ p+1\)
      \STATE \(start\gets\ start+1\)
    \ELSE
      \STATE \(values[start]\gets\ values[stop]\)
      \COMMENT{Insert the checked element between undiscovered and larger area}
      \STATE \(values[stop]\gets\ seen\)
      \STATE \(stop\gets\ stop-1\)
      \COMMENT{Shift the border between undiscovered and larger area}
    \ENDIF
  \ENDWHILE
  \RETURN\(p\)

\end{algorithmic}
\end{algorithm}


\floatname{algorithm}{Function}
\begin{algorithm}
\caption{testSorted(numbers)}
\begin{algorithmic}[5]

\STATE \COMMENT{ Checks whether or not the passed-in array is (ascendingly) sorted. }
  \STATE \(isSorted\gets\TRUE\)
  \STATE \(i\gets\ 0\)
  \WHILE{\(isSorted\ \AND(i\leq\ length(numbers)-2)\)}[As we compare with the following element, we must stop at the penultimate index]
    \IF{\(numbers[i]>numbers[i+1]\)}[Is there an inversion?]
      \STATE \(isSorted\gets\FALSE\)
    \ELSE
      \STATE \(i\gets\ i+1\)
    \ENDIF
  \ENDWHILE
  \RETURN\(isSorted\)

\end{algorithmic}
\end{algorithm}


\floatname{algorithm}{Procedure}
\begin{algorithm}
\caption{buildMaxHeap(heap)}
\begin{algorithmic}[5]

\STATE \COMMENT{ Runs through the array heap and converts it to a max-heap }
\STATE \COMMENT{ in a bottom-up manner, i.e. starts above the "{}leaf"{} level }
\STATE \COMMENT{ (index \textgreater= length(heap) div 2) and goes then up towards }
\STATE \COMMENT{ the root. }
  \STATE \(lgth\gets\ length(heap)\)
  \FOR{\(k \gets lgth\oprdiv\ 2-1\) \TO \(0\) \textbf{by} \(-1\)}
    \STATE \(maxHeapify(heap,k,lgth)\)
  \ENDFOR

\end{algorithmic}
\end{algorithm}


\floatname{algorithm}{Procedure}
\begin{algorithm}
\caption{quickSort(values, start, stop)}
\begin{algorithmic}[5]

\STATE \COMMENT{ Recursively sorts a subrange of the given array '{}values\textasciiacute{}.  }
\STATE \COMMENT{ start is the first index of the subsequence to be sorted, }
\STATE \COMMENT{ stop is the index BEHIND the subsequence to be sorted. }
  \IF{\(stop\geq\ start+2\)}[At least 2 elements? (Less don'{}t make sense.)]
    \STATE \COMMENT{ Select a pivot element, be p its index. }
    \STATE \COMMENT{ (here: randomly chosen element out of start ... stop-1) }
    \STATE \(p\gets\ random(stop-start)+start\)
    \STATE \COMMENT{ Partition the array into smaller and greater elements }
    \STATE \COMMENT{ Get the resulting (and final) position of the pivot element }
    \STATE \(p\gets\ partition(values,start,stop,p)\)
    \STATE \textbf{parallel} \COMMENT{Sort subsequances separately and independently ...} \BODY
      \STATE \textbf{thread} 1 \BODY
        \STATE \(quickSort(values,start,p)\)
        \COMMENT{Sort left (lower) array part}
      \ENDBODY \STATE \textbf{end thread} 1
      \STATE \textbf{thread} 2 \BODY
        \STATE \(quickSort(values,p+1,stop)\)
        \COMMENT{Sort right (higher) array part}
      \ENDBODY \STATE \textbf{end thread} 2
    \ENDBODY \STATE \textbf{end parallel}
  \ENDIF

\end{algorithmic}
\end{algorithm}


\floatname{algorithm}{Procedure}
\begin{algorithm}
\caption{heapSort(values)}
\begin{algorithmic}[5]

\STATE \COMMENT{ Sorts the array '{}values\textasciiacute{} of numbers according to he heap sort }
\STATE \COMMENT{ algorithm }
  \STATE \(buildMaxHeap(values)\)
  \STATE \(heapRange\gets\ length(values)\)
  \FOR{\(k \gets heapRange-1\) \TO \(1\) \textbf{by} \(-1\)}
    \STATE \(heapRange\gets\ heapRange-1\)
    \STATE \(maximum\gets\ values[0]\)
    \COMMENT{Swap the maximum value (root of the heap) to the heap end}
    \STATE \(values[0]\gets\ values[heapRange]\)
    \STATE \(values[heapRange]\gets\ maximum\)
    \STATE \(maxHeapify(values,0,heapRange)\)
  \ENDFOR

\end{algorithmic}
\end{algorithm}


\floatname{algorithm}{Program}
\begin{algorithm}
\caption{SORTING\_TEST\_MAIN()}
\begin{algorithmic}[5]

\STATE \COMMENT{ Creates three equal arrays of numbers and has them sorted with different sorting algorithms }
\STATE \COMMENT{ to allow performance comparison via execution counting ("{}Collect Runtime Data"{} should }
\STATE \COMMENT{ sensibly be switched on). }
\STATE \COMMENT{ Requested input data are: Number of elements (size) and filing mode. }
  \REPEAT
    \STATE\ \textbf{input}\ \(elementCount\)
  \UNTIL{\(elementCount\geq\ 1\)}
  \REPEAT
    \STATE\ \textbf{input}\ \(\)"{}Filling:\ 1\ =\ random,\ 2\ =\ increasing,\ 3\ =\ decreasing"{}\(,modus\)
  \UNTIL{\(modus=1\ \OR\ modus=2\ \OR\ modus=3\)}
  \FOR{\(i \gets 0\) \TO \(elementCount-1\) \textbf{by} \(1\)}
    \IF{\(modus=1\)}
      \STATE \(values1[i]\gets\ random(10000)\)
    \ELSIF{\(modus=2\)}
      \STATE \(values1[i]\gets\ i\)
    \ELSIF{\(modus=3\)}
      \STATE \(values1[i]\gets-i\)
    \ENDIF
  \ENDFOR
  \FOR{\(i \gets 0\) \TO \(elementCount-1\) \textbf{by} \(1\)}
    \STATE \(values2[i]\gets\ values1[i]\)
    \STATE \(values3[i]\gets\ values1[i]\)
  \ENDFOR
  \STATE \textbf{parallel}  \BODY
    \STATE \textbf{thread} 1 \BODY
      \STATE \(bubbleSort(values1)\)
    \ENDBODY \STATE \textbf{end thread} 1
    \STATE \textbf{thread} 2 \BODY
      \STATE \(quickSort(values2,0,elementCount)\)
    \ENDBODY \STATE \textbf{end thread} 2
    \STATE \textbf{thread} 3 \BODY
      \STATE \(heapSort(values3)\)
    \ENDBODY \STATE \textbf{end thread} 3
  \ENDBODY \STATE \textbf{end parallel}
  \STATE \(ok1\gets\ testSorted(values1)\)
  \STATE \(ok2\gets\ testSorted(values2)\)
  \STATE \(ok3\gets\ testSorted(values3)\)
  \IF{\(\NOT\ ok1\ \OR\ \NOT\ ok2\ \OR\ \NOT\ ok3\)}
    \FOR{\(i \gets 0\) \TO \(elementCount-1\) \textbf{by} \(1\)}
      \IF{\(values1[i]\neq\ values2[i]\ \OR\ values1[i]\neq\ values3[i]\)}
        \PRINT\(\)"{}Difference\ at\ ["{}\(,i,\)"{}]:\ "{}\(,values1[i],\)"{}\ \textless-\textgreater\ "{}\(,values2[i],\)"{}\ \textless-\textgreater\ "{}\(,values3[i]\)
      \ENDIF
    \ENDFOR
  \ENDIF
  \REPEAT
    \STATE\ \textbf{input}\ \(\)"{}Show\ arrays\ (yes/no)?"{}\(,show\)
  \UNTIL{\(show=\)"{}yes"{}\(\ \OR\ show=\)"{}no"{}\(\)}
  \IF{\(show=\)"{}yes"{}\(\)}
    \FOR{\(i \gets 0\) \TO \(elementCount-1\) \textbf{by} \(1\)}
      \PRINT\(\)"{}["{}\(,i,\)"{}]:\textbackslash{}t"{}\(,values1[i],\)"{}\textbackslash{}t"{}\(,values2[i],\)"{}\textbackslash{}t"{}\(,values3[i]\)
    \ENDFOR
  \ENDIF

\end{algorithmic}
\end{algorithm}

\end{document}
