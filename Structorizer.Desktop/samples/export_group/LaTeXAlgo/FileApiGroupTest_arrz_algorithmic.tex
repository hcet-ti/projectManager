\documentclass[a4paper,10pt]{article}

\usepackage{algorithm}
\usepackage{algorithmic}
\usepackage{ngerman}
\usepackage{amsmath}

\DeclareMathOperator{\oprdiv}{div}
\DeclareMathOperator{\oprshl}{shl}
\DeclareMathOperator{\oprshr}{shr}
\title{Structorizer LaTeX pseudocode Export of FileApiGroupTest.arrz}
\author{Kay G"urtzig}
\date{09.06.2021}

\begin{document}

\floatname{algorithm}{Program}
\begin{algorithm}
\caption{ComputeSum()}
\begin{algorithmic}[5]

\STATE \COMMENT{ Computes the sum and average of the numbers read from a user-specified }
\STATE \COMMENT{ text file (which might have been created via generateRandomNumberFile(4)). }
\STATE \COMMENT{  }
\STATE \COMMENT{ This program is part of an arrangement used to test group code export (issue }
\STATE \COMMENT{ \#828) with FileAPI dependency. }
\STATE \COMMENT{ The input check loop has been disabled (replaced by a simple unchecked input }
\STATE \COMMENT{ instruction) in order to test the effect of indirect FileAPI dependency (only the }
\STATE \COMMENT{ called subroutine directly requires FileAPI now). }
  \STATE \(fileNo\gets\ 1000\)
  \STATE\ \textbf{input}\ \(\)"{}Name/path\ of\ the\ number\ file"{}\(file\_name\)
  \COMMENT{Disable this if you enable the loop below!}
  \IF{\(fileNo>0\)}
    \STATE \(values\gets\{\}\)
    \STATE \(nValues\gets\ 0\)
    \STATE \textbf{try}  \BODY
      \STATE \(nValues\gets\ readNumbers(file\_name,values,1000)\)
    \ENDBODY \STATE \textbf{end try}
    \STATE \textbf{catch} (\(failure\)) \BODY
      \PRINT\(failure\)
      \STATE \textbf{exit} \(-7\)
    \ENDBODY \STATE \textbf{end catch}
    \STATE \(sum\gets\ 0.0\)
    \FOR{\(k \gets 0\) \TO \(nValues-1\) \textbf{by} \(1\)}
      \STATE \(sum\gets\ sum+values[k]\)
    \ENDFOR
    \PRINT\(\)"{}sum\ =\ "{}\(,sum\)
    \PRINT\(\)"{}average\ =\ "{}\(,sum/nValues\)
  \ENDIF

\end{algorithmic}
\end{algorithm}


\floatname{algorithm}{Program}
\begin{algorithm}
\caption{DrawRandomHistogram()}
\begin{algorithmic}[5]

\STATE \COMMENT{ Reads a random number file and draws a histogram accotrding to the }
\STATE \COMMENT{ user specifications }
  \STATE \(fileNo\gets-10\)
  \REPEAT
    \STATE\ \textbf{input}\ \(\)"{}Name/path\ of\ the\ number\ file"{}\(file\_name\)
    \STATE \(fileNo\gets\ fileOpen(file\_name)\)
  \UNTIL{\(fileNo>0\ \OR\ file\_name=\)"{}"{}\(\)}
  \IF{\(fileNo>0\)}
    \STATE \(fileClose(fileNo)\)
    \STATE\ \textbf{input}\ \(\)"{}number\ of\ intervals"{}\(,nIntervals\)
    \FOR{\(k \gets 0\) \TO \(nIntervals-1\) \textbf{by} \(1\)}
      \STATE \(count[k]\gets\ 0\)
    \ENDFOR
    \STATE \(kMaxCount\gets\ 0\)
    \COMMENT{Index of the most populated interval}
    \STATE \(numberArray\gets\{\}\)
    \STATE \(nObtained\gets\ 0\)
    \STATE \textbf{try}  \BODY
      \STATE \(nObtained\gets\ readNumbers(file\_name,numberArray,10000)\)
    \ENDBODY \STATE \textbf{end try}
    \STATE \textbf{catch} (\(failure\)) \BODY
      \PRINT\(failure\)
    \ENDBODY \STATE \textbf{end catch}
    \IF{\(nObtained>0\)}
      \STATE \(min\gets\ numberArray[0]\)
      \STATE \(max\gets\ numberArray[0]\)
      \FOR{\(i \gets 1\) \TO \(nObtained-1\) \textbf{by} \(1\)}
        \IF{\(numberArray[i]<min\)}
          \STATE \(min\gets\ numberArray[i]\)
        \ELSE
          \IF{\(numberArray[i]>max\)}
            \STATE \(max\gets\ numberArray[i]\)
          \ENDIF
        \ENDIF
      \ENDFOR
      \STATE \(width\gets(max-min)*1.0/nIntervals\)
      \COMMENT{Interval width}
      \FOR{\(i \gets 0\) \TO \(nObtained-1\) \textbf{by} \(1\)}
        \STATE \(value\gets\ numberArray[i]\)
        \STATE \(k\gets\ 1\)
        \WHILE{\(k<nIntervals\ \AND\ value>min+k*width\)}
          \STATE \(k\gets\ k+1\)
        \ENDWHILE
        \STATE \(count[k-1]\gets\ count[k-1]+1\)
        \IF{\(count[k-1]>count[kMaxCount]\)}
          \STATE \(kMaxCount\gets\ k-1\)
        \ENDIF
      \ENDFOR
      \STATE \(drawBarChart(count,nIntervals)\)
      \PRINT\(\)"{}Interval\ with\ max\ count:\ "{}\(,kMaxCount,\)"{}\ ("{}\(,count[kMaxCount],\)"{})"{}\(\)
      \FOR{\(k \gets 0\) \TO \(nIntervals-1\) \textbf{by} \(1\)}
        \PRINT\(count[k],\)"{}\ numbers\ in\ interval\ "{}\(,k,\)"{}\ ("{}\(,min+k*width,\)"{}\ ...\ "{}\(,min+(k+1)*width,\)"{})"{}\(\)
      \ENDFOR
    \ELSE
      \PRINT\(\)"{}No\ numbers\ read."{}\(\)
    \ENDIF
  \ENDIF

\end{algorithmic}
\end{algorithm}


\floatname{algorithm}{Procedure}
\begin{algorithm}
\caption{drawBarChart(values, nValues)}
\begin{algorithmic}[5]

\STATE \COMMENT{ Draws a bar chart from the array "{}values"{} of size nValues. }
\STATE \COMMENT{ Turtleizer must be activated and will scale the chart into a square of }
\STATE \COMMENT{ 500 x 500 pixels }
\STATE \COMMENT{ Note: The function is not robust against empty array or totally equal values. }
  \STATE \(const\ xSize\gets\ 500\)
  \COMMENT{Used range of the Turtleizer screen}
  \STATE \(const\ ySize\gets\ 500\)
  \STATE \(kMin\gets\ 0\)
  \STATE \(kMax\gets\ 0\)
  \FOR{\(k \gets 1\) \TO \(nValues-1\) \textbf{by} \(1\)}
    \IF{\(values[k]>values[kMax]\)}
      \STATE \(kMax\gets\ k\)
    \ELSE
      \IF{\(values[k]<values[kMin]\)}
        \STATE \(kMin\gets\ k\)
      \ENDIF
    \ENDIF
  \ENDFOR
  \STATE \(valMin\gets\ values[kMin]\)
  \STATE \(valMax\gets\ values[kMax]\)
  \STATE \(yScale\gets\ valMax*1.0/(ySize-1)\)
  \STATE \(yAxis\gets\ ySize-1\)
  \IF{\(valMin<0\)}
    \IF{\(valMax>0\)}
      \STATE \(yAxis\gets\ valMax*ySize*1.0/(valMax-valMin)\)
      \STATE \(yScale\gets(valMax-valMin)*1.0/(ySize-1)\)
    \ELSE
      \STATE \(yAxis\gets\ 1\)
      \STATE \(yScale\gets\ valMin*1.0/(ySize-1)\)
    \ENDIF
  \ENDIF
  \STATE \(gotoXY(1,ySize-1)\)
  \COMMENT{draw coordinate axes}
  \STATE \(forward(ySize-1)\)
  \STATE \(penUp()\)
  \STATE \(backward(yAxis)\)
  \STATE \(right(90)\)
  \STATE \(penDown()\)
  \STATE \(forward(xSize-1)\)
  \STATE \(penUp()\)
  \STATE \(backward(xSize-1)\)
  \STATE \(stripeWidth\gets\ xSize/nValues\)
  \FOR{\(k \gets 0\) \TO \(nValues-1\) \textbf{by} \(1\)}
    \STATE \(stripeHeight\gets\ values[k]*1.0/yScale\)
    \STATE \(discr607e3da0 <- k mod 3\)
    \IF{\(discr607e3da0=0\)}
      \STATE \(setPenColor(255,0,0)\)
    \ELSIF{\(discr607e3da0=1\)}
      \STATE \(setPenColor(0,255,0)\)
    \ELSIF{\(discr607e3da0=2\)}
      \STATE \(setPenColor(0,0,255)\)
    \ENDIF
    \STATE \(fd(1)\)
    \STATE \(left(90)\)
    \STATE \(penDown()\)
    \STATE \(fd(stripeHeight)\)
    \STATE \(right(90)\)
    \STATE \(fd(stripeWidth-1)\)
    \STATE \(right(90)\)
    \STATE \(forward(stripeHeight)\)
    \STATE \(left(90)\)
    \STATE \(penUp()\)
  \ENDFOR

\end{algorithmic}
\end{algorithm}


\floatname{algorithm}{Function}
\begin{algorithm}
\caption{generateRandomNumberFile(fileName, count, minVal, maxVal)}
\begin{algorithmic}[5]

\STATE \COMMENT{ Writes a text file with name fileName, consisting of count lines, each containing }
\STATE \COMMENT{ a random number in the range from minVal to maxVal (both inclusive). }
\STATE \COMMENT{ Returns either the number of written number if all went well or some potential }
\STATE \COMMENT{ error code if something went wrong. }
  \STATE \(fileNo\gets\ fileCreate(fileName)\)
  \IF{\(fileNo\leq\ 0\)}
    \RETURN\(fileNo\)
  \ENDIF
  \STATE \textbf{try}  \BODY
    \FOR{\(k \gets 1\) \TO \(count\) \textbf{by} \(1\)}
      \STATE \(number\gets\ minVal+random(maxVal-minVal+1)\)
      \STATE \(fileWriteLine(fileNo,number)\)
    \ENDFOR
  \ENDBODY \STATE \textbf{end try}
  \STATE \textbf{catch} (\(error\)) \BODY
    \PRINT\(error\)
    \RETURN\(-7\)
  \ENDBODY \STATE \textbf{end catch}
  \STATE \textbf{finally} \BODY
    \STATE \(fileClose(fileNo)\)
  \ENDBODY \STATE \textbf{end finally}
  \RETURN\(count\)

\end{algorithmic}
\end{algorithm}


\floatname{algorithm}{Function}
\begin{algorithm}
\caption{readNumbers(fileName, numbers, maxNumbers)}
\begin{algorithmic}[5]

\STATE \COMMENT{ Tries to read as many integer values as possible upto maxNumbers }
\STATE \COMMENT{ from file fileName into the given array numbers. }
\STATE \COMMENT{ Returns the number of the actually read numbers. May cause an exception. }
  \STATE \(nNumbers\gets\ 0\)
  \STATE \(fileNo\gets\ fileOpen(fileName)\)
  \IF{\(fileNo\leq\ 0\)}
    \STATE \textbf{throw} \(\)"{}File could not be opened!"{}\(\)
  \ENDIF
  \STATE \textbf{try}  \BODY
    \WHILE{\(\NOT\ fileEOF(fileNo)\ \AND\ nNumbers<maxNumbers\)}
      \STATE \(number\gets\ fileReadInt(fileNo)\)
      \STATE \(numbers[nNumbers]\gets\ number\)
      \STATE \(nNumbers\gets\ nNumbers+1\)
    \ENDWHILE
  \ENDBODY \STATE \textbf{end try}
  \STATE \textbf{catch} (\(error\)) \BODY
    \STATE \textbf{throw} \(\)
  \ENDBODY \STATE \textbf{end catch}
  \STATE \textbf{finally} \BODY
    \STATE \(fileClose(fileNo)\)
  \ENDBODY \STATE \textbf{end finally}
  \RETURN\(nNumbers\)

\end{algorithmic}
\end{algorithm}

\end{document}
